%%%%%%%%%%%%%%%%%%%%%%%%%%%%%%%%%%%%%%%%%
% American Geophysical Union (AGU)
% LaTeX Template
% Version 1.0 (3/6/13)
%
% This template has been downloaded from:
% http://www.LaTeXTemplates.com
%
% Original author:
% The AGUTeX class and agu-ps referencing style were created and are owned 
% by AGU: http://publications.agu.org/author-resource-center/author-guide/latex-formatting-toolkit/
%
% This template has been modified from the blank AGU template to include
% examples of how to insert content and drastically change commenting. The
% structural integrity is maintained as in the original blank template.
%
% Important notes: 
% This template retains extensive commenting from the AGU template. It is heavily 
% advised you read these comments and follow them in order to insure a speedy 
% submission process.
%
%%%%%%%%%%%%%%%%%%%%%%%%%%%%%%%%%%%%%%%%%

%%%%%%%%%%%%%%%%%%%%%%%%%%%%%%%%%%%%%%%%%%%%%%%%%%%%%%%%%%%%%%%%%%%%%%%%%%%%
% AGUtmpl.tex: this template file is for articles formatted with LaTeX2e,
% Modified March 2013
%
% This template includes commands and instructions
% given in the order necessary to produce a final output that will
% satisfy AGU requirements.
%
% PLEASE DO NOT USE YOUR OWN MACROS
% DO NOT USE \newcommand, \renewcommand, or \def.
%
% FOR FIGURES, DO NOT USE \psfrag or \subfigure.
%
%%%%%%%%%%%%%%%%%%%%%%%%%%%%%%%%%%%%%%%%%%%%%%%%%%%%%%%%%%%%%%%%%%%%%%%%%%%%
%
% All questions should be e-mailed to latex@agu.org.
%
%%%%%%%%%%%%%%%%%%%%%%%%%%%%%%%%%%%%%%%%%%%%%%%%%%%%%%%%%%%%%%%%%%%%%%%%%%%%

% Step 1: Set the \documentclass

% There are two options for article format: two column (default) and draft.

% PLEASE USE THE DRAFT OPTION TO SUBMIT YOUR PAPERS.
% The draft option produces double spaced output.

% Choose the journal abbreviation for the journal you are submitting to:

% jgrga	JOURNAL OF GEOPHYSICAL RESEARCH
% gbc	GLOBAL BIOCHEMICAL CYCLES
% grl		GEOPHYSICAL RESEARCH LETTERS
% pal	PALEOCEANOGRAPHY
% ras	RADIO SCIENCE
% rog	REVIEWS OF GEOPHYSICS
% tec	TECTONICS
% wrr	WATER RESOURCES RESEARCH
% gc		GEOCHEMISTRY, GEOPHYSICS, GEOSYSTEMS
% sw	SPACE WEATHER
% ms	JAMES
%
%
%
% (If you are submitting to a journal other than jgrga,
% substitute the initials of the journal for "jgrga" below.)

\documentclass[draft,jgrga]{AGUTeX}

% To create numbered lines:

% If you don't already have lineno.sty, you can download it from http://www.ctan.org/tex-archive/macros/latex/contrib/ednotes/ (or search the internet for lineno.sty ctan), available at TeX Archive Network (CTAN). Take care that you always use the latest version.

% To activate the commands, uncomment \usepackage{lineno} and \linenumbers*[1]command, below:

%\usepackage{lineno}
%\linenumbers*[1]

%  To add line numbers to lines with equations:
%  \begin{linenomath*}
%  \begin{equation}
%  \end{equation}
%  \end{linenomath*}

%%%%%%%%%%%%%%%%%%%%%%%%%%%%%%%%%%%%%%%%%%%%%%%%%%%%%%%%%%%%%%%%%%%%%%%%%
% Figures and Tables

% DO NOT USE \psfrag or \subfigure commands.

%  Figures and tables should be placed AT THE END OF THE ARTICLE, after the references.

%  Uncomment the following command to include .eps files (comment out this line for draft format):
%\usepackage[dvips]{graphicx}
\usepackage{graphicx}

% Substitute one of the following for [dvips] above if you are using a different driver program and want to proof your illustrations on your machine:
% [xdvi], [dvipdf], [dvipsone], [dviwindo], [emtex], [dviwin],
% [pctexps],  [pctexwin],  [pctexhp],  [pctex32], [truetex], [tcidvi],
% [oztex], [textures]

%  Uncomment the following command to allow illustrations to print when using Draft:
\setkeys{Gin}{draft=false}

% See how to enter figures and tables at the end of the article, after references.

%----------------------------------------------------------------------------------------
%	RUNNING HEAD AND CORRESPONDING AUTHOR
%----------------------------------------------------------------------------------------

% Author names in capital letters:
\authorrunninghead{SMITH ET AL.}

%------------------------------------------------

% Shorter version of title entered in capital letters:
\titlerunninghead{SHORT TITLE}

%------------------------------------------------

% Corresponding author mailing address and e-mail address:
\authoraddr{Corresponding author: Jane Smith, Department of Geography, Ohio State University, Columbus, Ohio, USA. (j.smith@ohio.edu)}

%----------------------------------------------------------------------------------------

\begin{document}

%----------------------------------------------------------------------------------------
%	TITLE
%----------------------------------------------------------------------------------------

\title{Study on the impact of applying soil organic content profile on soil physics and surface fluxes}

%----------------------------------------------------------------------------------------
%	AUTHORS AND AFFILIATIONS
%----------------------------------------------------------------------------------------

% Use \author{\altaffilmark{}} and \altaffiltext{}

% \altaffilmark will produce footnote; matching \altaffiltext will appear at bottom of page.

\authors{Shijie Shu,\altaffilmark{1}
Umakant Mishra,\altaffilmark{2}, and
Atul Jain\altaffilmark{1}}

\altaffiltext{1}{Department of Atmospheric Sciences, University of Illinois at Urbana Champaign, Tucson, Arizona, USA.}

\altaffiltext{2}{Department of Geography, Ohio State University, Columbus, Ohio, USA.}

%----------------------------------------------------------------------------------------
%	ABSTRACT
%----------------------------------------------------------------------------------------

% Do NOT include any \begin...\end commands within the body of the abstract.

\begin{abstract}

Lorem ipsum dolor sit amet, consectetur adipiscing elit. Sed vehicula metus sapien. Suspendisse pulvinar, felis ut hendrerit aliquet, dui nisi bibendum erat, fermentum mattis enim nibh id arcu. Vestibulum ultrices eros sed odio tincidunt bibendum. Pellentesque fermentum ante vel nulla commodo fermentum. Vestibulum in augue sit amet libero viverra accumsan eu at magna. Sed at ligula quis nibh pharetra facilisis non eu libero. Suspendisse non quam sit amet massa luctus interdum sit amet in purus. Integer id orci elit, vitae sollicitudin lectus.

\end{abstract}

%----------------------------------------------------------------------------------------
%	ARTICLE CONTENT
%----------------------------------------------------------------------------------------

% The body of the article must start with a \begin{article} command
% \end{article} must follow the references section, before the figures and tables.

\begin{article}

\section{Introduction}
To understand soil organic carbon (SOC) storage is important since its high potential on impacting global carbon emission. Currently LSMs are still applying simple models in estimating carbon storage, while the SOC dynamic in northern high latitude is more complicated in physical and geomorphological perspectives. One most important thing currently missed in the model is the feedback on energy and moisture redistribution in soil through the evolving of SOC profile. With linking model estimated SOC content to the SOC profile, the changing profile has a huge potential on impacting energy and water redistribution in the soil which will further stabalize/destabilize the SOC stored in the soil since soil organic carbon content has important impact on inlfuencing soil thermal and hydraulic properties. However, limited works (Koven et al., 2009) have been done on studying this potential feedback.\\
A number of studies have shown decreased soil bulk density and increased water holding capacity, porosity, aggregation and biological activity following the application of farm manures. Soil organic matter is closely related to aggregate stability (Tisdall and Oades, 1982; Elliott, 1986) and soil erodibility (Kay, 2000). Soil bulk density and porosity are functions of soil organic matter, aggregate stability and size distribution, and soil particle density (Mapa, 1995; Baldock and Nelson, 2000). A decrease in organic matter would cause an increase in bulk density and a decrease in porosity, thereby reducing soil infiltration, and water and air storage capacities (Franzluebbers, 2002; Wall and Heiskanen, 2003; Celik, 2005).\\

We note that, although CLM4 does prognostically calculate soil carbon content, for these simulations the simulated soil carbon content is not used to set the thermal and hydrologic soil properties because of an extreme low bias in simulated Arctic soil carbon stocks in the CLM4 soil biogeochemistry model. The low soil carbon bias is likely at least partially due to the lack of a representation in CLM of critical soil processes that govern the accumulation of soil organic material in the cold, moist high-latitude climate regime such as decomposition constraints due to anoxia and the mixing of organic matter into the soil through cryoturbation (Koven et al. 2009).\\

Consequently, the conditions that determine soil carbon cycling are highly depth-dependent and different mechanisms may be operating in different layers within one profile (Rumpel et al., 2002; Salome et al. ? , 2010; Rumpel and Kogel-Knabner, 2011). Therefore, aggregation of properties and processes over the profile, as is currently done in most SOM models (e.g. Parton et al., 1987; Schimel et al., 1994), is likely an oversimplification, inadequate to support new parameterizations of relevant processes.\\

Thus, additional information is required in order to parameterize dynamic SOM profile models. In past studies, 13C and 14C, have been used as tracers to this purpose (Elzein and Balesdent, 1995; Freier et al., 2010; Baisden et al., 2002). Although these isotopes are particularly useful for constraining organic matter turnover times and carbon pathways, their precise information content with respect to the processes involved in SOM profile formation is less clear, since rhizodeposition leads to direct input of 13C and 14C at depth.\\

Fallout radio-isotopes (e.g. 137Cs, 134Cs, 210Pbex, 7Be)\\
Such tracers have two major advantages over carbon isotopes: (i) loss occurs only due to radioactive decay, which is constant and exactly known; and (ii) input occurs only at the soil surface - direct input at depth is negligible.\\
These points imply that the vertical transport rate of such isotopes can be directly inferred from their concentration profile (Kaste et al., 2007; He and Walling, 1997).\\

Use 14C to validate the profile we calculated? (HOW TO?)\\

The 13C/12C and 14C/12C ratios were determined on all samples. The 14C/12C ratios were expressed with respect to the 0.95\% U.S. National Bureau of Standards (NBS) oxalic acid standard and were then normalized for the isotopic fractionation effect with respect to the 13C/12C ratio of the Peedee Belemite standard (PDB).\\

When fresh organic matter is returned to the soil, the radiocarbon is diluted by the older soil C so that the "bomb" enrichment is less marked than in the herbage.\\

In deeper soil horizons, organic material must move down the profile, either by direct physical translocation or by the burrowing of earthworms. It also continues to the diluted as it moves downwards.  If it is considered to be a diffusion process, whether simply physical or due to the activity of earthworms, the movement is a function of the square of the distance and a function of time, and enrichment by diffusion is much less, the greater the depth (Prokhorov, 1975).\\

The change in value with depth (Lower Delta 13C at the top soil) reflects the preferential loss of the lighter isotope, 12C , during animal and microbial metabolism and this pattern is typical of normal well-drained soils (Stout and Rafter, 1978).\\

Parameters: Carbon input (Litter), Decomposition rate (including respiration and leaching), Diffusivity of C through the soil (Bioturbation) and Turnover time (Why we need this?).\\

However the experiments conducted by Jenkinson (1965) with radiocarbon-enriched ryegrass closely simulated the conditions in the field. His experiments showed a rapid exponential loss of C in the field, with some 80\% disappearing either by respiration or leaching in the first year but with the remaining 20\% disappearing at a very much slower rate. In the model this variable is indicated by \lamda.\\
 
O'Brien 1978 used dispersion model to estimate 13C and 14C profile and estimate parameters.\\
Bruun 2007 applied both diffusive and advective terms to model carbon isotope profile. Parameters have been determined by fitting the model to the observations according to the method of least squares by optimizing D and v.\\
Jarvis 2010 applied local and non-local mixing, use both dispersive and advective motions and separate 137Cs into solution and sorbed part in the soil. The dispersion and advection through both water (Leaching) and sorbed soil particles (Bioturbation) have been modeled plus spontaneous radioactive dacay for 137Cs and non-local transfer through egestion of soil animals (mainly earthworm).\\

Globally, soil organic matter (SOM) contains 1500 Pg C to 1 m soil depth (Post et al. 1982) and 2300 Pg C to 3 m depth (Jobbagy and Jackson 2000) -- more than biomass and atmospheric CO2 combined.\\
It has been assumed that SOC changes related to land-use change occur almost entirely in the upper $30 cm$ of soil, because deep SOC is inert on timescales <100 years and is transported slowly downward relative to plant sources (Baisden et al. 2002a; O'Brien and Stout 1978).\\

Our approach is designed to test for and quantify the amount of reactive SOC below $30 cm$ depth (henceforth referred to as deep RSOC) that cannot be considered to have residence times of centuries or millennia, and therefore can be expected to react to global changes.\\

The reasons for these very variable results of parameter values are probably differences in soil type, precipitation regime and faunal activity.\\

Despite the existence previous models discussed in section 1.2.6, there are several distinct gaps in research related to SOM profile modelling:\\
Most SOM profile models were intended to complement measurements in order to explain observed concentration profiles of soil organic carbon, radiocarbon,or other constituents. As such they typically do not account for effects of soil temperature and moisture.\\
None of the SOM profile models represents both the organic layer and the mineral soil profile.\\
Typically all vertical SOM transport is lumped into one transport term, diffusion or advection. Several models do include both formulations but generally no effort is made to relate them to specific processes in the field such as bioturbation and liquid phase transport.\\
Very few of the published studies present thorough assessment of parameter uncertainty and its effect on the reliability of model predictions. Furthermore, the value of different observations to estimate parameters is not discussed.\\

Formation, development, maintenance of such ground patterns and their interaction with vegetation is poorly understood [Walker et al., 2004].\\
We instrumented several nonsorted circles at these sites with sensors that measure soil temperature, moisture content and the maximum frost heave [Walker et al., 2004].\\

The net vertical flux of SOC is thus proportional to the depth gradient of SOC concentrations. Because of this characteristic, mixing?driven net SOC transport has been called diffusive transport [Elzein and Balesdent, 1995]. Whereas this approach is legitimate for understanding the development of SOC profiles that are affected only by net SOC fluxes, it is gross SOC fluxes that reflect the dynamics of SOC mixing. For example, vigorous gross fluxes, when added up, may cancel out such that the resulting net flux becomes negligible. The degree of SOC dynamics by mixing can be thus correctly assessed by considering gross fluxes rather than net fluxes.\\

If net SOC flux was used in determining mixing?driven turnover time, the turnover time would be ~17 years. By considering only gross fluxes, we correctly estimate that SOC is replaced by mixing in the Ap2 horizon in 1.4 ��0.8 years.\\

"Ideas".\\
1. What makes soil organic carbon being stabalized? (Large question)\\

2. What are the actual processes happened related to SOC dynamics in order to model it? (Decomposition and movement)\\
(1) Root exudation, microbial activity and priming. (Need to check Kritina's protocal and Foatain's paper?)\\
(2) Bioturbation/Cryoturbation.\\
    Bioturbation: Actually it's an upside and down process. Mainly happened at horizon A and AB where it's more significant than disolved organic carbon flux.\\
    Cryoturbation: Complicated process. Four models have been proposed to describe cryoturbation, and the cryoturbation happened mostly between active layer. The common sense is the cryoturbation has a larger strength for bringing SOC downward comparing to bioturbation. As Charles Koven's method he applied a logrithm decaying curve to describe the cryoturbation strength below the active layer, however whether this looks reasonable or not needed to be examined.\\
    - subquestion 1: Anyway we can represent horizon and its change in the model? The main purpose for doing this is to provide a clue on parameterizing diffusive rate and advective rate seperately since in Horizon A diffusive processes dominated while in Horizon B advective happened. Cryoturbation is an exception.\\

(3) Dissolved Organic Carbon Dynamics and Illuviation/Eluviation of SOC. How it related to the soil water? (Need more literature review)\\

(4) Exoenzymatic activity to convert SOC from polymers to soluble monomers.\\

(5) Stabilization mechanism of SOC in mineral soil.\\
   - Adsorption of dissolved organic carbon on minerals.\\
   - Soil Chelation Mechanism preserving SOC from microbes?\\
   - Soil humic substances have long been comfused to be a macromolecules exist in soil are actually simple bimolecules. This raises the potential destabilization mechanism.\\
   - Mineralogical control on soil organic carbon. Long residence-time carbon pool, mainly minerally-stabilized carbon, determine the accumulation and loss of soil organic carbon storage. (Torn et al., 1997)\\

3. What kind of relationship do plant have on soil forming and transporting? \\

4. About the AGU talk. Water storage change under future impact? How to seperate the contribution of water and temperature to the SOC storgae under future senario?\\

Questions:
1. Temperture impact on soil respiration and soil organic carbon content distribution impacting temperature redistribution.\\
2. Soil moisture impacting soil respiration and vertical mixing through leaching and the redistributed soil organic carbon changed redistribution of soil moisture. \\
3. How the contribution of soil respiration from the deep SOC contribute in non-permafrost region?\\
4. How these interact with future climate? Do we need to seperate the impact of temperature and soil mosture to study it? If we need to do that, how?\\
5. WHat can I get from Umakant except data?\\


********************************
What direction can we go currently to model decomposition on a vertically-resolved based SOM model?\\
Care mostly about all that maybe important to temperature feedback.\\

It will be good to look at seperated organic and mineral horizons. (Yi et al., 2009, GRL)\\
Soil depth can follow the equation 2 from He et al., 2014 paper's supplementary information. Or ask about how to estimate the horizon depth?\\
In different horizons what different kind of mechanisms it will have on impacting the SOC diffusion, advection and decomposition? Estimate something?\\


%%Nam fermentum sapien at enim varius consectetur. Quisque lobortis imperdiet mauris, et accumsan libero vulputate vitae. Integer lacinia purus vel metus tempus suscipit. Curabitur ac sapien quis mauris euismod commodo. Sed pharetra sem elit. Fusce ultrices, mauris eu fermentum tempor, tellus sem ornare lectus, in convallis nunc urna id dolor. Donec convallis ligula vitae sem viverra fermentum. Mauris in ullamcorper erat. Donec ultrices tempus nibh quis vestibulum. This statement requires citation \citep{AtkinsonSloan}. This one is an in-text citation because the authors of \citet{ColtonKress1} are specifically mentioned.
%%
%%Praesent volutpat, nibh in dignissim commodo, tellus justo consequat erat, vel consequat mi arcu vel lectus. Aliquam a tellus nec felis sagittis consequat. Quisque convallis imperdiet neque a tempor. Nulla non erat urna. Mauris vel lorem magna, tristique auctor ipsum. Aliquam pharetra eleifend massa. Donec porttitor sagittis luctus. Aliquam pretium luctus leo quis congue. Morbi vel felis mi. Suspendisse viverra tortor pretium orci lacinia eleifend. Phasellus aliquam, nunc eu cursus feugiat, erat odio porttitor libero, quis accumsan orci ipsum ut lorem. Vestibulum pharetra malesuada egestas. Sed non orci sit amet erat suscipit fringilla in et diam. Vestibulum ante ipsum primis in faucibus orci luctus et ultrices posuere cubilia Curae; Nunc ut rhoncus nulla. Aenean porta rhoncus suscipit.
%%
%------------------------------------------------

\section{Materials and Methods}

ISAM is a land surface model applied in several international studies already. Here we extend the soil bigeochemistry part of ISAM model following these several steps: (1) Soil organic carbon model has been seperated into several different layers. (2) Vertical mixing of soil organic matter has been treated through applying convection-diffusion equation. (3) Decomposition is considered at each soil layer. (4) Soil thermal and hydraulic conductivities are linked to soil organic matter density to consider the feedback effect.
Within these midifications, three mechanisms have been highligted here: cryoturbation processes have been seperately modeled through applying a larger number .
Temperature change?
CHanges in hydrology?
thawing of permafrost, and changes in hydrology,


\begin{equation}
\label{eq:emc}
e = mc^2
\end{equation}

\begin{eqnarray}
  x_{1} & = & (x - x_{0}) \cos \Theta \nonumber \\
        && + (y - y_{0}) \sin \Theta  \nonumber \\
  y_{1} & = & -(x - x_{0}) \sin \Theta \nonumber \\
        && + (y - y_{0}) \cos \Theta.
\end{eqnarray}

%------------------------------------------------

\section{Results}

Referencing equation \ref{eq:emc}. Mauris vel lorem magna, tristique auctor ipsum. Aliquam pharetra eleifend massa. Donec porttitor sagittis luctus. Aliquam pretium luctus leo quis congue. Morbi vel felis mi. Referencing Table \ref{sampletable}. Referencing Figure \ref{placeholder}.

\subsection{Simulations}

\subsubsection{Simulation 1}

Vivamus magna enim, aliquet id cursus a, pharetra ut purus. Phasellus suscipit nisi iaculis mi vulputate id interdum velit dictum. Nam ullamcorper elit in lectus ultrices vitae volutpat massa gravida. Etiam sagittis commodo neque eget placerat. Sed et nisi faucibus metus interdum adipiscing id nec lacus. Donec ipsum diam, malesuada at euismod consectetur, placerat quis diam. Phasellus cursus semper viverra. Proin magna tortor, blandit in ultricies id, facilisis at nibh. Proin eu neque est. Etiam euismod auctor ante. Mauris mauris sem, tincidunt a placerat rutrum, porta id est. Aenean non velit porta eros condimentum facilisis at in nibh. Etiam cursus purus ut orci rhoncus sit amet semper eros porttitor. Etiam ac leo at ipsum tincidunt consequat ac non sapien. Aenean sed leo diam, venenatis pharetra odio.

\subsubsection{Simulation 2}

Suspendisse viverra eleifend nulla at facilisis. Nullam eget tellus orci. Cras sit amet lorem velit. Maecenas rhoncus pellentesque orci eget vulputate. Phasellus massa nisi, mattis nec elementum accumsan, blandit non neque. In ac enim elit, sit amet luctus ante. Cras feugiat commodo lectus, vitae convallis dui sagittis id. In in tellus lacus, sed lobortis eros. Phasellus sit amet eleifend velit. Duis ornare dapibus porttitor. Maecenas eros velit, dignissim at egestas in, tincidunt lacinia erat. Proin elementum mi vel lectus suscipit fringilla. Mauris justo est, ullamcorper in rutrum interdum, accumsan eget mi. Maecenas ut massa aliquet purus eleifend vehicula in a nisi. Fusce molestie cursus lacinia.

\subsection{Real Data}

Aliquam interdum pellentesque scelerisque. Sed tincidunt suscipit purus, id aliquet nulla vehicula quis. Duis sed nisl lorem. Vivamus erat ante, dignissim et aliquam vel, adipiscing vitae magna. Cras id dapibus metus. Cum sociis natoque penatibus et magnis dis parturient montes, nascetur ridiculus mus. Proin ut lectus ut nisi congue ullamcorper. Ut ac turpis ligula. Sed faucibus bibendum nunc eget gravida.

%------------------------------------------------

\section{Discussion}

Nam fermentum sapien at enim varius consectetur. Quisque lobortis imperdiet mauris, et accumsan libero vulputate vitae. Integer lacinia purus vel metus tempus suscipit. Curabitur ac sapien quis mauris euismod commodo. Sed pharetra sem elit. Fusce ultrices, mauris eu fermentum tempor, tellus sem ornare lectus, in convallis nunc urna id dolor. Donec convallis ligula vitae sem viverra fermentum. Mauris in ullamcorper erat. Donec ultrices tempus nibh quis vestibulum.

Praesent volutpat, nibh in dignissim commodo, tellus justo consequat erat, vel consequat mi arcu vel lectus. Aliquam a tellus nec felis sagittis consequat. Quisque convallis imperdiet neque a tempor. Nulla non erat urna. Mauris vel lorem magna, tristique auctor ipsum. Aliquam pharetra eleifend massa. Donec porttitor sagittis luctus. Aliquam pretium luctus leo quis congue. Morbi vel felis mi. Suspendisse viverra tortor pretium orci lacinia eleifend. Phasellus aliquam, nunc eu cursus feugiat, erat odio porttitor libero, quis accumsan orci ipsum ut lorem. Vestibulum pharetra malesuada egestas. Sed non orci sit amet erat suscipit fringilla in et diam. Vestibulum ante ipsum primis in faucibus orci luctus et ultrices posuere cubilia Curae; Nunc ut rhoncus nulla. Aenean porta rhoncus suscipit.

Vivamus magna enim, aliquet id cursus a, pharetra ut purus. Phasellus suscipit nisi iaculis mi vulputate id interdum velit dictum. Nam ullamcorper elit in lectus ultrices vitae volutpat massa gravida. Etiam sagittis commodo neque eget placerat. Sed et nisi faucibus metus interdum adipiscing id nec lacus. Donec ipsum diam, malesuada at euismod consectetur, placerat quis diam. Phasellus cursus semper viverra. Proin magna tortor, blandit in ultricies id, facilisis at nibh. Proin eu neque est. Etiam euismod auctor ante. Mauris mauris sem, tincidunt a placerat rutrum, porta id est. Aenean non velit porta eros condimentum facilisis at in nibh. Etiam cursus purus ut orci rhoncus sit amet semper eros porttitor. Etiam ac leo at ipsum tincidunt consequat ac non sapien. Aenean sed leo diam, venenatis pharetra odio.

Suspendisse viverra eleifend nulla at facilisis. Nullam eget tellus orci. Cras sit amet lorem velit. Maecenas rhoncus pellentesque orci eget vulputate. Phasellus massa nisi, mattis nec elementum accumsan, blandit non neque. In ac enim elit, sit amet luctus ante. Cras feugiat commodo lectus, vitae convallis dui sagittis id. In in tellus lacus, sed lobortis eros. Phasellus sit amet eleifend velit. Duis ornare dapibus porttitor. Maecenas eros velit, dignissim at egestas in, tincidunt lacinia erat. Proin elementum mi vel lectus suscipit fringilla. Mauris justo est, ullamcorper in rutrum interdum, accumsan eget mi. Maecenas ut massa aliquet purus eleifend vehicula in a nisi. Fusce molestie cursus lacinia.

%----------------------------------------------------------------------------------------
%	APPENDICES (OPTIONAL)
%----------------------------------------------------------------------------------------

%%%%%%%%%%%%%%%%%%%%%%%%%%%%%%%%
%% Optional Appendix goes here

% \appendix resets counters and redefines section heads
% but doesn't print anything.
% After typing  \appendix

% \section{Here Is Appendix Title}
% will show
% Appendix A: Here Is Appendix Title

\appendix

\section{Appendix Title}

Vivamus magna enim, aliquet id cursus a, pharetra ut purus. Phasellus suscipit nisi iaculis mi vulputate id interdum velit dictum. Nam ullamcorper elit in lectus ultrices vitae volutpat massa gravida. Etiam sagittis commodo neque eget placerat. Sed et nisi faucibus metus interdum adipiscing id nec lacus. Donec ipsum diam, malesuada at euismod consectetur, placerat quis diam. Phasellus cursus semper viverra. Proin magna tortor, blandit in ultricies id, facilisis at nibh. Proin eu neque est. Etiam euismod auctor ante. Mauris mauris sem, tincidunt a placerat rutrum, porta id est. Aenean non velit porta eros condimentum facilisis at in nibh. Etiam cursus purus ut orci rhoncus sit amet semper eros porttitor. Etiam ac leo at ipsum tincidunt consequat ac non sapien. Aenean sed leo diam, venenatis pharetra odio.

%----------------------------------------------------------------------------------------
%	GLOSSARY OR NOTATION (OPTIONAL)
%----------------------------------------------------------------------------------------

%%%%%%%%%%%%%%%%%%%%%%%%%%%%%%%%%%%%%%%%%%%%%%%%%%%%%%%%%%%%%%%%
%
% Optional Glossary or Notation section, goes here
%
%%%%%%%%%%%%%%
% Glossary is only allowed in Reviews of Geophysics
% \section*{Glossary}
% \paragraph{Term}
% Term Definition here
%
%%%%%%%%%%%%%%
% Notation -- End each entry with a period.
% \begin{notation}
% Term & definition.\\
% Second term & second definition.\\
% \end{notation}
%%%%%%%%%%%%%%%%%%%%%%%%%%%%%%%%%%%%%%%%%%%%%%%%%%%%%%%%%%%%%%%%

%----------------------------------------------------------------------------------------
%	ACKNOWLEDGEMENTS
%----------------------------------------------------------------------------------------

\begin{acknowledgments}
This work was partially supported by a grant from the Spanish Ministry of Science and Technology.
\end{acknowledgments}

%----------------------------------------------------------------------------------------
%	BIBLIOGRAPHY
%----------------------------------------------------------------------------------------

% Either type in your references using
% \begin{thebibliography}{}
% \bibitem{}
% Text
% \end{thebibliography}

% Or,

% If you use BiBTeX for your references, please use the agufull08.bst file (available at % ftp://ftp.agu.org/journals/latex/journals/Manuscript-Preparation/) to produce your .bbl
% file and copy the contents into your paper here.

% Follow these steps:
% 1. Run LaTeX on your LaTeX file.

% 2. Make sure the bibliography style appears as \bibliographystyle{agufull08}. Run BiBTeX on your LaTeX
% file.

% 3. Open the new .bbl file containing the reference list and
%   copy all the contents into your LaTeX file here.

% 4. Comment out the old \bibliographystyle and \bibliography commands.

% 5. Run LaTeX on your new file before submitting.

% AGU does not want a .bib or a .bbl file. Please copy in the contents of your .bbl file here.

\begin{thebibliography}{}

\providecommand{\natexlab}[1]{#1}
\expandafter\ifx\csname urlstyle\endcsname\relax
  \providecommand{\doi}[1]{doi:\discretionary{}{}{}#1}\else
  \providecommand{\doi}{doi:\discretionary{}{}{}\begingroup
  \urlstyle{rm}\Url}\fi

\bibitem[{\textit{Atkinson and Sloan}(1991)}]{AtkinsonSloan}
Atkinson, K., and I.~Sloan (1991), The numerical solution of first-kind
  logarithmic-kernel integral equations on smooth open arcs, \textit{Math.
  Comp.}, \textit{56}(193), 119--139.

\bibitem[{\textit{Colton and Kress}(1983)}]{ColtonKress1}
Colton, D., and R.~Kress (1983), \textit{Integral Equation Methods in
  Scattering Theory}, John Wiley, New York.

\bibitem[{\textit{Hsiao et~al.}(1991)\textit{Hsiao, Stephan, and
  Wendland}}]{StephanHsiao}
Hsiao, G.~C., E.~P. Stephan, and W.~L. Wendland (1991), On the {D}irichlet
  problem in elasticity for a domain exterior to an arc, \textit{J. Comput.
  Appl. Math.}, \textit{34}(1), 1--19.

\bibitem[{\textit{Lu and Ando}(2012)}]{LuAndo}
Lu, P., and M.~Ando (2012), Difference of scattering geometrical optics
  components and line integrals of currents in modified edge representation,
  \textit{Radio Sci.}, \textit{47},  RS3007, \doi{10.1029/2011RS004899}.

\end{thebibliography}

% Reference citation examples:

%...as shown by \textit{Kilby} [2008].
%...as shown by {\textit  {Lewin}} [1976], {\textit  {Carson}} [1986], {\textit  {Bartholdy and Billi}} [2002], and {\textit  {Rinaldi}} [2003].
%...has been shown [\textit{Kilby et al.}, 2008].
%...has been shown [{\textit  {Lewin}}, 1976; {\textit  {Carson}}, 1986; {\textit  {Bartholdy and Billi}}, 2002; {\textit  {Rinaldi}}, 2003].
%...has been shown [e.g., {\textit  {Lewin}}, 1976; {\textit  {Carson}}, 1986; {\textit  {Bartholdy and Billi}}, 2002; {\textit  {Rinaldi}}, 2003].

%...as shown by \citet{jskilby}.
%...as shown by \citet{lewin76}, \citet{carson86}, \citet{bartoldy02}, and \citet{rinaldi03}.
%...has been shown \citep{jskilbye}.
%...has been shown \citep{lewin76,carson86,bartoldy02,rinaldi03}.
%...has been shown \citep [e.g.,][]{lewin76,carson86,bartoldy02,rinaldi03}.

% Please use ONLY \citet and \citep for reference citations.
% DO NOT use other cite commands (e.g., \cite, \citeyear, \nocite, \citealp, etc.).

\end{article}

%----------------------------------------------------------------------------------------
%	FIGURES AND TABLES
%----------------------------------------------------------------------------------------

%% Enter Figures and Tables here:
%
% DO NOT USE \psfrag or \subfigure commands.
%
% Figure captions go below the figure.
% Table titles go above tables; all other caption information should be placed in footnotes below the table.
%
%----------------
% EXAMPLE FIGURE
%
% \begin{figure}
% \noindent\includegraphics[width=20pc]{samplefigure.eps}
% \caption{Caption text here}
% \label{figure_label}
% \end{figure}
%
% ---------------
% EXAMPLE TABLE
%
%\begin{table}
%\caption{Time of the Transition Between Phase 1 and Phase 2\tablenotemark{a}}
%\centering
%\begin{tabular}{l c}
%\hline
% Run  & Time (min)  \\
%\hline
%  $l1$  & 260   \\
%  $l2$  & 300   \\
%  $l3$  & 340   \\
%  $h1$  & 270   \\
%  $h2$  & 250   \\
%  $h3$  & 380   \\
%  $r1$  & 370   \\
%  $r2$  & 390   \\
%\hline
%\end{tabular}
%\tablenotetext{a}{Footnote text here.}
%\end{table}

% See below for how to make sideways figures or tables.

\newpage

\begin{figure}
\includegraphics[width=0.4\linewidth]{placeholder.jpg}
\caption{Figure caption}\label{placeholder}
\end{figure}

\begin{table}
\caption{Table caption}\label{sampletable}
\begin{tabular}{l l l}
\hline
\textbf{Treatments} & \textbf{Response 1} & \textbf{Response 2}\\
\hline
Treatment 1 & 0.0003262 & 0.562 \\
Treatment 2 & 0.0015681 & 0.910 \\
Treatment 3 & 0.0009271 & 0.296 \\
\hline
\end{tabular}
\end{table}

\end{document}

%%%%%%%%%%%%%%%%%%%%%%%%%%%%%%%%%%%%%%%%%%%%%%%%%%%%%%%%%%%%%%%

More Information and Advice:

%% ------------------------------------------------------------------------ %%
%
%  SECTION HEADS
%
%% ------------------------------------------------------------------------ %%

% Capitalize the first letter of each word (except for
% prepositions, conjunctions, and articles that are
% three or fewer letters).

% AGU follows standard outline style; therefore, there cannot be a section 1 without
% a section 2, or a section 2.3.1 without a section 2.3.2.
% Please make sure your section numbers are balanced.
% ---------------
% Level 1 head
%
% Use the \section{} command to identify level 1 heads;
% type the appropriate head wording between the curly
% brackets, as shown below.
%
%An example:
%\section{Level 1 Head: Introduction}
%
% ---------------
% Level 2 head
%
% Use the \subsection{} command to identify level 2 heads.
%An example:
%\subsection{Level 2 Head}
%
% ---------------
% Level 3 head
%
% Use the \subsubsection{} command to identify level 3 heads
%An example:
%\subsubsection{Level 3 Head}
%
%---------------
% Level 4 head
%
% Use the \subsubsubsection{} command to identify level 3 heads
% An example:
%\subsubsubsection{Level 4 Head} An example.
%
%% ------------------------------------------------------------------------ %%
%
%  IN-TEXT LISTS
%
%% ------------------------------------------------------------------------ %%
%
% Do not use bulleted lists; enumerated lists are okay.
% \begin{enumerate}
% \item
% \item
% \item
% \end{enumerate}
%
%% ------------------------------------------------------------------------ %%
%
%  EQUATIONS
%
%% ------------------------------------------------------------------------ %%

% Single-line equations are centered.
% Equation arrays will appear left-aligned.

Math coded inside display math mode \[ ...\]
 will not be numbered, e.g.,:
 \[ x^2=y^2 + z^2\]

 Math coded inside \begin{equation} and \end{equation} will
 be automatically numbered, e.g.,:
 \begin{equation}
 x^2=y^2 + z^2
 \end{equation}

% IF YOU HAVE MULTI-LINE EQUATIONS, PLEASE
% BREAK THE EQUATIONS INTO TWO OR MORE LINES
% OF SINGLE COLUMN WIDTH (20 pc, 8.3 cm)
% using double backslashes (\\).

% To create multiline equations, use the
% \begin{eqnarray} and \end{eqnarray} environment
% as demonstrated below.
\begin{eqnarray}
  x_{1} & = & (x - x_{0}) \cos \Theta \nonumber \\
        && + (y - y_{0}) \sin \Theta  \nonumber \\
  y_{1} & = & -(x - x_{0}) \sin \Theta \nonumber \\
        && + (y - y_{0}) \cos \Theta.
\end{eqnarray}

%If you don't want an equation number, use the star form:
%\begin{eqnarray*}...\end{eqnarray*}

% Break each line at a sign of operation
% (+, -, etc.) if possible, with the sign of operation
% on the new line.

% Indent second and subsequent lines to align with the first character following the equal sign on the first line.

% Use an \hspace{} command to insert horizontal space into your equation if necessary. Place an appropriate unit of measure between the curly braces, e.g. \hspace{1in}; you may have to experiment to achieve the correct amount of space.


%% ------------------------------------------------------------------------ %%
%
%  EQUATION NUMBERING: COUNTER
%
%% ------------------------------------------------------------------------ %%

% You may change equation numbering by resetting
% the equation counter or by explicitly numbering
% an equation.

% To explicitly number an equation, type \eqnum{}
% (with the desired number between the brackets)
% after the \begin{equation} or \begin{eqnarray}
% command.  The \eqnum{} command will affect only
% the equation it appears with; LaTeX will number
% any equations appearing later in the manuscript
% according to the equation counter.
%

% If you have a multiline equation that needs only
% one equation number, use a \nonumber command in
% front of the double backslashes (\\) as shown in
% the multiline equation above.

%% ------------------------------------------------------------------------ %%
%
%  SIDEWAYS FIGURE AND TABLE EXAMPLES
%
%% ------------------------------------------------------------------------ %%
%
% For tables and figures, add \usepackage{rotating} to the paper and add the rotating.sty file to the folder.
% AGU prefers the use of {sidewaystable} over {landscapetable} as it causes fewer problems.
%
% \begin{sidewaysfigure}
% \includegraphics[width=20pc]{samplefigure.eps}
% \caption{caption here}
% \label{label_here}
% \end{sidewaysfigure}
%
% \begin{sidewaystable}
% \caption{}
% \begin{tabular}
% Table layout here.
% \end{tabular}
% \end{sidewaystable}
